%!TEX root = ../Main.tex
\section{Everything is Roses}

\TODO{Give some examples of trees}

\tikzset{
    treenode/.style = {
        circle
      , font = \sffamily
      , align = center
      , text centered
      , text width = 1.5em
      , inner sep = 0pt
      , very thick
      }
  , outcome/.style = {
        treenode
      , font = \sffamily\bfseries
      , white
      , draw = black
      , fill = black
      }
  , shrink/.style = {
        treenode
      , black
      , draw = black
      }
}

%
% I have no idea what I'm doing with this marginbox thing, it just looked weird
% without it - feel free to fix when the time comes.
%
% The nodes which have 'at (x,y)' below have slightly tweaked relative
% positions, as the default layout assumes a balanced tree, which shrink trees
% are not.
%
\marginbox{-0.5em 1em 0em 1em}{\begin{tikzpicture}[
    ->
  , > = stealth'
  , level/.style = {
        sibling distance = 3.9cm/#1
      , level distance = 1cm
      }
  ]
\node [outcome] {5}
  child {
    node [shrink] at (2,0) {0}
  }
  child {
    node [shrink] {3}
    child {
      node [shrink] {0}
    }
    child {
      node [shrink] at (-1,0) {2}
      child {
        node [shrink] {0}
      }
      child {
        node [shrink] {1}
        child {
          node [shrink] {0}
        }
      }
    }
  }
  child {
    node [shrink] at (-1.65,0) {4}
    child {
      node [shrink] at (1,0) {0}
    }
    child {
      node [shrink] {2}
      child {
        node [shrink] {0}
      }
      child {
        node [shrink] {1}
        child {
          node [shrink] {0}
        }
      }
    }
    child {
      node [shrink] at (0.3,0) {3}
      child {
        node [shrink] {0}
      }
      child {
        node [shrink] {2}
        child {
          node [shrink] {0}
        }
        child {
          node [shrink] {1}
          child {
            node [shrink] {0}
          }
        }
      }
    }
  }
;
\end{tikzpicture}}

\TODO{Not sure what text would be good here? I've knocked up the shrink tree for the integer 5.}

\begin{code}
  data Tree a =
    Node a [Tree a]
\end{code}

QuickCheck implements shrinking for properties using a rose tree. A property is just a generator which produces a tree of results. The root of the tree is the result of the randomly generated test case, while the shrinks are the sub-trees.

\begin{code}
  data Result =
    Result {
        success         :: Bool
      , counterexamples :: [String]
      }

  newtype Property =
    Property (Gen (Tree Result))
\end{code}

Lazy trees offer a very clean way to search the space of simplified test cases, but only exposing its construction to the programmer via a type class bound @shrink@ function has a number of problems.

While generators can be combined using the familiar @Functor@ / @Applicative@ / @Monad@ interface, @Arbitrary@ instances cannot. One might think that changing @Arbitrary@ to a data type might solve this problem, but unfortunately the @shrink@ function is invariant, and thus @Arbitrary@ can never be a covariant @Functor@.

\begin{code}
  data Arbitrary a =
    Arbitrary {
        arbitrary :: Gen a
      , shrink    :: a -> [a] -- invariant
      }
\end{code}

We propose that generators themselves should produce a tree. Not only does this allow generators to implement more involved shrinking strategies, the covariant nature of rose trees allows us to keep the convenient @Functor@ / @Applicative@ / @Monad@ interface which we know and love.

\TODO{Flesh this out a bit more?}

\begin{code}
  newtype Gen a =
    Gen (StdGen -> Tree a)
\end{code}

Imagine a general system for storing well-typed data, with built-in aggregation capabilities. Such a system might be used in the context of feature engineering, for a machine learning platform, such as the one we have developed at Ambiata.

In order to test the functions involved in implementing this system, one would need to generate sets of values which have compatible schemas. This is incredibly difficult using a type directed approach to generation and shrinking.

\TODO{Not sure if the aggregation part is more than necessary at this point. The fact that the values are aggregates means they are semigroups, so we can talk about merging/appending them together. It also gives us something which is in the schema, but not available in the value, so we can't derive the schema from the value.}

\begin{code}
  data Aggregate =
      Minimum
    | Maximum
    | Sum

  data Schema =
      SInt Int Int Aggregate
    | SPair Schema Schema

  data Value =
      VInt Int
    | VPair Value Value
\end{code}

If we wanted to test a function which requires two sets of values with matching schemas, the obvious thing to do is to generate a schema, then use that to generate values which comply with the schema. To do this, we will use a number of QuickCheck's @Gen@ combinators. @elements@ selects a random value from a list of options, @oneof@ executes a random generator from the input list and @choose@ generates a value in the specified range.

\begin{code}
  elements :: [a] -> Gen a

  oneof :: [Gen a] -> Gen a

  choose :: Random a => (a, a) -> Gen a
\end{code}

Implementing an @Arbitrary@ instance for this scenario requires the creation of a new data type, and the implementation of the @shrink@ function for this data type is anything but obvious.

\begin{code}
  data TwoValue =
    TwoValue Schema [Value] [Value]

  genSInt :: Gen Schema
  genSInt = do
    x <- arbitrary
    y <- arbitrary
    SInt (min x y) (max x y)
      <$> elements [Minimum, Maximum, Sum]

  genSchema :: Gen Schema
  genSchema =
    oneof [
        genSInt
      , SPair <$> genSchema <*> genSchema
      ]

  genValue :: Schema -> Gen Value
  genValue = \case
    SInt vmin vmax _ ->
      VInt <$> choose (vmin, vmax)
    SPair sx sy ->
      VPair <$> genValue sx <*> genValue sy

  instance Arbitrary TwoValue where
    arbitrary = do
      schema <- genSchema
      TwoValue schema
        <$> listOf (genValue schema)
        <*> listOf (genValue schema)
    shrink =
      -- ??? TODO should I actually write this out?

  prop_test :: TwoValue -> Property
  prop_test (TwoValue schema vs0 vs1) =
    -- some test
\end{code}

Because writing shrinkers is difficult, programmers often forgo this step entirely when writing @Arbitrary@ instances. Opting instead to use the default implementation which just returns the empty list. In some cases they may be able to use @genericShrink@, which uses @GHC.Generics@ to automatically derive a shrink function for their data type. This will not, however, take in to account invariants which a data type may have. \TODO{hackage / github results here?}

In our example, the schema @SInt@ specifies the range of values a @VInt@ can take. If we tried to use @genericShrink@ here it would happily shrink to a value which broke the invariants imposed by the schema.

Another problem with the QuickCheck version of our example is that it uses @arbitrary@ to generate @Int@ values for the @SInt@ constructor. This seems like the obvious thing to do, but the @Arbitrary@ instance for @Int@ will only generate values from @0@ to @100@. This may be fine for many situations, but in our case we'd like to test the full range of @Int@. This can easily be done in QuickCheck, but the type-directed approach to generating values entices programmers towards the wrong solution here. \TODO{this implicit decision has actually caused me a lot of pain, with generators not doing what I thought they were doing, not sure how to say this though. I've also found coworkers doing this and not realising that they weren't testing what they thought they were.}

With rose tree generators we can solve these problems. Because we need to provide shrinking as well as input generation, some of our combinators are slightly different to QuickCheck. While QuickCheck has a single @choose@ combinator for generating values in a range, we have a number of different ones. This is because we need to know more about a type in order to provide a shrinker for it.

\begin{code}
  integral :: Integral a => a -> a -> Gen a

  realFloat :: RealFloat a => a -> a -> Gen a

  enum :: Enum a => a -> a -> Gen a
\end{code}

Here are the same set of generators constructed using our approach.

\begin{code}
  genSInt :: Gen Schema
  genSInt = do
    x <- integral minBound maxBound
    y <- integral minBound maxBound
    SInt (min x y) (max x y)
      <$> elements [Minimum, Maximum, Sum]

  genSchema :: Gen Schema
  genSchema =
    oneof [
        genSInt
      , SPair <$> genSchema <*> genSchema
      ]

  genValue :: Schema -> Gen Value
  genValue = \case
    SInt vmin vmax _ ->
      VInt <$> integral vmin vmax
    SPair sx sy ->
      VPair <$> genValue sx <*> genValue sy

  prop_test :: Property ()
  prop_test = do
    schema <- forAll genSchema
    vs0 <- forAll $ listOf (genValue schema)
    vs1 <- forAll $ listOf (genValue schema)
    -- some test
\end{code}

When it comes to building up generators, the essence isn't a lot of different. However, we haven't needed to construct a data type solely for the purpose of making an @Arbitrary@ instance. In addition, because we've escewed our overloaded @arbitrary@ function for explicit combinators, it's obvious that we're testing the whole range of @Int@ when we construct an @SInt@. The key difference though, is that we haven't had to think about writing a shrink function, it comes for free, simply by combining generators. Not only that, the shrinks for our values are always guaranteed to obey our schema, by construction.
