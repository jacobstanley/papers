%!TEX root = ../Main.tex
\section{Everything is Roses}

Introduce the main ideas here, but leave \emph{added extras} for the next section. The rose tree data structure, examples are at the first level of the tree, shrinks are lower levels.

Here is our new definition of the @Arbitrary@ class.

\begin{code}
class Arbitrary a where
   arbitrary :: Gen (Tree a)
\end{code}

Our generator now produces a candidate value and a lazy tree of all its possible reductions, rather than just a single candidate at a time. Give the data type definition for @Tree@. Give some examples of trees.

Talk about a key example, describe how it works and introduce parts of our new system along the way.

Need a key example which has a bug, but not an "obvious bug". At the same time it shouldn't take too long to explain.

Or: the key example could focus on generating data that obeys some schema, and then the shrunk examples still need to match the schema.