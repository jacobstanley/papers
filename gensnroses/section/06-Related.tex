%!TEX root = ../Main.tex
\clearpage
\section{Related Work}

Sentence about QuickCheck\cite{claessen:quickcheck}. 

Cite all the related work that we have found and \emph{give a comparison} as to how that work is different to our own. If ours is better then say how.

We suspect that some other systems are implemented in this way, but it's not written up properly. Suspect that the commercial Erlang version of QuickCheck works in this way. 

There is also a Closure library that uses a rose tree, but it's not written up. Suspect that the bind implementation is wrong. Bind re-splits the tree for every shrink. Don't want to re-generate new random values when we shrink, just less of the old values. Example, when shrinking a 10 elem list the result might have 5 elems of the prev list, but not 5 entirely new elements. 

Python library has integrated shrinking but is not implemented with a rose tree.

