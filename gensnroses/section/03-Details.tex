%!TEX root = ../Main.tex
\section{Details}

QuickCheck has the notion of a \emph{size} parameter, which is varied between test case executions. Smaller sizes are tried first, before gradually increasing the size as further test cases are executed. This helps to produce smaller counterexamples as trivial inputs are tested before more complex ones. It also provides a way for the programmer to avoid diverging when generating recursive data structures. As such, our generators also accept a size parameter when generating values. \TODO{Discussed in original QuickCheck paper: 3.2 Generators for User-Defined Types}

\begin{code}
  newtype Size =
    Size Int

  newtype Gen a =
    Gen (Size -> StdGen -> Rose a)
\end{code}

Concretely, it may help to think of the size parameter as a percentage which never gets to @100@. When the test runner executes the first test case for a property, the size is @0@. The size then increases by one for each test until it gets to @99@ on the 100th test, before reseting to @0@ on the 101st test.

%
% Size vs Number of Tests
%

\begin{tikzpicture}
\begin{axis}[
    xlabel = Number of Tests
  , xtick = {1, 100, 200}
  , ylabel = Size
  , ytick = {0, 50, 99}
  , ylabel style = {
        at = { (0.05, 0.5) }
      }
  ]
\addplot[plot-blue, thick] plot coordinates {
  (1,0)
  (100,99)
  (101,0)
  (200,99)
};
\end{axis}
\end{tikzpicture}

Some of QuickCheck's combinators, such as @listOf@, make use of the size parameter to limit the values they generate. While others, such as @choose@ are unaffected. Unsurprisingly, none of QuickCheck's combinators offer control over how a value shrinks, because shrinking in QuickCheck is handled with the @Arbitrary@ type class.

We provide a range of combinators for controlling both how a value shrinks, and how it is affected by the size parameter. The workhorse for these combinators is the @Range@ data type.

\begin{code}
  data Range a =
    Range a (Size -> (a, a))
\end{code}

A @Range@ consists of an \emph{origin}, which is the value we would like to shrink towards, and a \emph{bounds function}, which calculates the upper and lower bound for a generator, given the current size. On top of this range abstraction, we can build a number of combinators which allow us finer control over the scope of generated values.

\begin{code}
  singleton :: a -> Range a
  singleton x =
    Range x (const (x, x))
\end{code}

@singleton@ constructs a range which always generates the same value. \TODO{not sure if 'singleton' is even worth mentioning? it does give a feel for how Range works I suppose.}

\begin{code}
  constant :: a -> a -> Range a
  constant x y =
    constantFrom x x y

  constantFrom :: a -> a -> a -> Range a
  constantFrom z x y =
    Range z (const (x, y))
\end{code}

The constant range combinators are unaffected by the size parameter. @constant@ constructs a range which shrinks to the first bound specified. Note that the bounds do not need to be specified as low then high, @constant 10 0@ is a perfectly legal application which means "generate a value between 0 and 10, shrinking towards 10". @constantFrom@ gives slightly more control, it takes an extra argument which specifies the origin to shrink towards.

%-------------------------------------------------------------------------------
% Plots for constant/constantFrom

%
% Range.constant 0 10
%
\begin{tikzpicture}
\begin{axis}[
    title = constant 0 10
  , title style = {
        font=\ttfamily
      }
  , xtick = {0, 99}
  , xlabel = Size
  , xlabel style = {
        at = { (0.5, 0.05) }
      }
  , ytick = {0, 5, 10}
  , ylabel = Value
  , ylabel style = {
        at = { (0.05, 0.5) }
      }
  , ymin = -2
  , ymax = 12
  , legend style= {
        at = { (0.05, 0.5) }
      , anchor = west
      }
  ]

\addplot[plot-blue, thick] plot coordinates {
  (0,10)
  (99,10)
};
\addlegendentry{Upper Bound}

\addplot[plot-green, thick] plot coordinates {
  (0,0)
  (99,0)
};
\addlegendentry{Lower Bound}

\node (upper)  at (axis cs:99,10) {};
\node (origin) at (axis cs:99,0) {};
\draw[->, gray, thick, dashed] (upper) -- (origin);

\end{axis}
\end{tikzpicture}

%
% Range.constantFrom 0 (-100) 100
%
\begin{tikzpicture}
\begin{axis}[
    title = constantFrom 0 (-100) 100
  , title style = {
        font=\ttfamily
      }
  , xtick = {0, 99}
  , xlabel = Size
  , xlabel style = {
        at = { (0.5, 0.05) }
      }
  , ytick = {-100, 0, 100}
  , ylabel = Value
  , ylabel style = {
        at = { (0.05, 0.5) }
      }
  , legend style= {
        at = { (0.05, 0.71) }
      , anchor = west
      }
  ]

\addplot[plot-blue, thick] plot coordinates {
  (0,100)
  (99,100)
};
\addlegendentry{Upper Bound}

\addplot[plot-orange, thick] plot coordinates {
  (0,0)
  (99,0)
};
\addlegendentry{Origin}

\addplot[plot-green, thick] plot coordinates {
  (0,-100)
  (99,-100)
};
\addlegendentry{Lower Bound}

\node (upper)  at (axis cs:99,100) {};
\node (origin) at (axis cs:99,0) {};
\node (lower)  at (axis cs:99,-100) {};
\draw[->, gray, thick, dashed] (upper) -- (origin);
\draw[->, gray, thick, dashed] (lower) -- (origin);

\end{axis}
\end{tikzpicture}

% End plots for constant/constantFrom
%-------------------------------------------------------------------------------

\begin{code}
  linear :: Integral a => a -> a -> Range a
  linear x y =
    linearFrom x x y

  linearFrom :: Integral a => a -> a -> a -> Range a
  linearFrom z x y =
    Range z $ \size ->
      ( clamp x y (scaleLinear size z x)
      , clamp x y (scaleLinear size z y) )

  -- Not sure if we should include the definition
  -- of clamp and scaleLinear here?
\end{code}

The linear range combinators are similar to the constant range combinators except that their bounds scale, away from the origin, linearly with the size.

\begin{code}
  genSInt :: Gen Schema
  genSInt = do
    x <- integral $ linearFrom 0 minBound maxBound
    y <- integral $ linearFrom 0 minBound maxBound
    SInt (min x y) (max x y)
      <$> elements [Minimum, Maximum, Sum]
\end{code}

%-------------------------------------------------------------------------------
% Plots for linear/linearFrom

%
% Range.linear 32 1024
%
\begin{tikzpicture}
\begin{axis}[
    /pgf/number format/.cd
  , use comma
  , 1000 sep = {}
  , title = linear 32 1024
  , title style = {
        font=\ttfamily
      }
  , xlabel = Size
  , xtick = {0, 99}
  , xlabel style = {
        at = { (0.5, 0.05) }
      }
  , ylabel = Value
  , ytick = {32, 1024}
  , ylabel style = {
        at = { (0.05, 0.5) }
      }
  , legend style= {
        at = { (0.05, 0.95) }
      , anchor = north west
      }
  ]

\addplot[plot-blue, thick] plot coordinates {
  (0,32)
  (99,1024)
};
\addlegendentry{Upper Bound}

\addplot[plot-green, thick] plot coordinates {
  (0,32)
  (99,32)
};
\addlegendentry{Lower Bound}

\node (upper)  at (axis cs:99,1024) {};
\node (origin) at (axis cs:99,32) {};
\draw[->, gray, thick, dashed] (upper) -- (origin);

\end{axis}
\end{tikzpicture}

%
% Range.linearFrom 2000 1970 2100
%
\begin{tikzpicture}
\begin{axis}[
    /pgf/number format/.cd
  , use comma
  , 1000 sep = {}
  , title = linearFrom 2000 1970 2100
  , title style = {
        font=\ttfamily
      }
  , xtick = {0, 99}
  , xlabel = Size
  , xlabel style = {
        at = { (0.5, 0.05) }
      }
  , ytick = {1970, 2000, 2100}
  , ylabel = Value
  , ylabel style = {
        at = { (0.05, 0.5) }
      }
  , legend style= {
        at = { (0.05, 0.95) }
      , anchor = north west
      }
  ]

\addplot[plot-blue, thick] plot coordinates {
  (0,2000)
  (99,2100)
};
\addlegendentry{Upper Bound}

\addplot[plot-orange, thick] plot coordinates {
  (0,2000)
  (99,2000)
};
\addlegendentry{Origin}

\addplot[plot-green, thick] plot coordinates {
  (0,2000)
  (99,1970)
};
\addlegendentry{Lower Bound}

\node (upper)  at (axis cs:99,2100) {};
\node (origin) at (axis cs:99,2000) {};
\node (lower)  at (axis cs:99,1970) {};
\draw[->, gray, thick, dashed] (upper) -- (origin);
\draw[->, gray, thick, dashed] (lower) -- (origin);

\end{axis}
\end{tikzpicture}

% End plots for linear/linearFrom
%-------------------------------------------------------------------------------
